
\section{Introduction}

The \textit{Online Encyclopedia of Integer Sequences} \citep{OEIS} is an online
database of sequences of numbers that collects any kind of data regarding them.
It was founded by N.~J.~A.~Sloane in $1964$ and since then has been, and
continue to be, updated constantly by contributions of many users. Despite of
its powerful searching mechanisms, starting at \url{https://oeis.org/} and
shown in Figure \ref{fig:oeis:page} which reports results about the well known
sequence of Fibonacci numbers (see e.g.  \citep{ConcreteMath}), we design a
parallel \textit{suite of software tools} that satisfies the necessities (i)~to
search the OEIS offline by downloading repeated searches, (ii)~to work in the
console to use programming facilities to manipulate contents more effectively
and (iii)~to interface with third-party software.

\begin{figure}
\centering
\includegraphics{OEIS/oeis-page}
\caption{The OEIS search page: the present snapshot shows search
results about the sequence of Fibonacci numbers }
\label{fig:oeis:page}
\end{figure}

Looking for similar approaches in the recent literature,
\citep{Nguyen_miningthe} mines the OEIS for new mathematical identities,
discussing how to store, compare and match integer sequences toward the
formalization of some conjectures; on the other hand, searching the word
"\textit{oeis}" in GitHub returns one hundred repositories, the majority of
them host simple implementations of scripts that download data about a
given sequence, targeting all major programming languages. Moreover,
\citep{weidmann:sequencer} is a project that identifies number sequences given
a list of numbers and gives a formula that generates them.

Our approach complements the former ones because it provides a
\textit{recursive} and \textit{asynchronous} fetching process, vanilla data
storage in JSON files and visualization of sequences' relations; the
description of each tool is addressed in the following sections, respectively.

The present suite of tools had been shown at an open school on
Combinatorial Method in the analysis of Algorithms and Data Structures
in Korea \citep{Nocentini:korea}; moreover, all the sources that
implements the applications can be found online in the repository
\url{https://github.com/massimo-nocentini/oeis-tools}.

\section{The Crawler}

The script \verb|crawling.py| implements a bot that given a sequence identifier
in the form $Axxxxxx$, where $x$s are digits, issues an HTTP request to the
main OEIS server and waits for a response; once it is received, the bot stores
it locally and, looking into the response's \verb|xref| section that contains a
set of other sequences identifiers, repeats its behaviour on each one of them,
recursively.  Such a bot is commonly known as \emph{crawler}.

Our implementation features neither threads nor race conditions nor data sync;
on the contrary, it targets \textit{pure asynchronous computation} by using
\textit{async/await} Python primitives only. The approach is educational and we
strive to create a simple but elegant codebase which boils down to $300$ lines
of Python code; eventually, it allows us to cache portions of the OEIS to speed
up repeated lookups and to restart the fetching process from the cache already
downloaded.

The script presents a help message to explain itself:
\VerbatimInput[fontsize=\small]{OEIS/crawler-help.txt}

\begin{example}
We illustrate a typical session where we start from scratch. First of all, we
want to download the OEIS content about two important and nice sequences,
namely those corresponding to the Fibonacci and Catalan numbers (see e.g.
\citep{ConcreteMath,stanley:2015}), known in the OEIS by identifiers $A000045$ and
$A000108$, respectively:
\VerbatimInput[fontsize=\small]{OEIS/crawler-fetching-command.txt}
After stopping the crawler, we check the content of cache with the commands
\VerbatimInput[fontsize=\small]{OEIS/crawler-status.txt} which tell us that
$30$ sequences had been fetched and stored in the default directory
\verb|./fetched/|.  Moreover, we can restart the crawler from where it was
interrupted with the command
\VerbatimInput[fontsize=\small]{OEIS/crawler-restarting.txt}
and we check that new sequences are actually collected,
\begin{Verbatim}[fontsize=\small]
$ python3.6 crawling.py
50 sequences in cache ./fetched/
354 sequences in fringe for restarting
\end{Verbatim}
as desired.
\end{example}

Having contents stored in JSON files allows us to inspect and manipulate them
using every tool available in our working environment, as the next example
shows.

\begin{example}
Combining the \verb|cat| command with the Python module \verb|json.tool|, that prints JSON files
with respect to indentation, we can visualize data about the sequence of Fibonacci
numbers as follows
\VerbatimInput[fontsize=\small]{OEIS/crawler-A000045-chunk.txt}
\end{example}


Our implementation takes strong inspiration from
\citep{VANROSSUM:DAVIS:async:await} and provides the following main abstractions:

\begin{description}
\item[reader] objects, that have the responsibility to be \textit{asynchronous
iterators}, having to respond to the message \verb|__anext__| where the
computation waits asynchronously for incoming data from the \verb|self.read|
coroutine. The code
\inputminted[fontsize=\small,stripnl=false,firstline=28,lastline=39]
{python}{../src/crawling.py} implements the description precisely.

\item[fetcher] objects have the responsibilities to (i)~create a socket with
OEIS server, (ii)~establish a working connection, (iii)~send an HTTP \verb|GET|
request for the desired sequence, (iv)~wait for the fetching process completes
and (v)~close the socket and signal that the work ends successfully.
\inputminted[fontsize=\small,stripnl=false,firstline=41,lastline=86]
    {python}{../src/crawling.py}

\item[crawler] objects have the responsibilities (i)~to keep a queue of task,
one for each candidate sequence, (ii)~to put each ready task into the
scheduling process and (iii)~to reclaim memory for completed task and,
eventually, (iv)~to deque them.
\inputminted[fontsize=\small,stripnl=false,firstline=89,lastline=117]
    {python}{../src/crawling.py}

\end{description}

\iffalse
Finally, the function \verb|oeis| puts all together and it is the main
interface exported by the \verb|crawling| module:
\inputminted[fontsize=\small,stripnl=false,firstline=195,lastline=221]
    {python}{../src/crawling.py}
\fi

\section{The (Pretty) Printer}

The script \verb|pprinting.py| provides a proxy for searching into the OEIS,
therefore it shows exactly the same contents you see from usual web interface
on \url{http://oeis.org}; additionally, it provides (i)~tabular representations
of \verb|data| sections in \textit{one and two dimensions} using list and
matrix notations, respectively, (ii)~filtering capabilities on most response's
sections and (iii)~interoperability with the crawler tool by taking advantage of
cached sequences.

The script presents a help message to explain itself:
\VerbatimInput[fontsize=\small]{OEIS/pprinting-help.txt} In the next examples
we show how \verb|pprinting|'s facilities can be used to apply filters, to
print data-only visualization and to search by an open query, respectively.

\begin{example} Typing the following command into a shell, it outputs
on the \verb|stdout| the pretty-printed contents about the sequence of
Fibonacci numbers, with two filters applied that show comments made by
prof. Barry and the first $5$ formulae only,
\VerbatimInput[fontsize=\small]{OEIS/pprinting-A000045.txt}
other sections, such as \verb|reference| and \verb|link|, are hidden by
default to provide a cleaner output.
\end{example}

\begin{example}
The following command pretty prints (i)~the first 3 sequences from our current
cache --the result may vary if you try on your own machine--, (ii)~ranking them
according to the most recent access time, (iii)~reporting data only and
(iv)~limiting up to $10$ coefficients for linear sequences:
\VerbatimInput[fontsize=\small]{OEIS/pprinting-data-only.txt}
\end{example}

\begin{example}
The following command pretty prints (i)~response about the open query
"\verb|pascal triangle|", (ii)~using $2$-dimension representation for matrices
in \verb|data| sections and (iii)~reports the first $2$ sequences in the
returned list only,
\VerbatimInput[fontsize=\small]{OEIS/pprinting-pascal-matrix.txt}
\end{example}

In parallel of the textual interface, we develop pretty printing functions that
integrates in Jupyter notebooks. The aim remains the same, namely to present
contents taken from the OEIS targeting a different environment that accepts
their representation; this is the time of a dynamic web interface that allows
us to evaluate Python code on the fly. Using Jupyter's Markdown language to
write textual content, we propose another view of the same data, as shown in
Figures \ref{fig:oeis:notebook:fibonacci}, \ref{fig:oeis:notebook:catalan} and
\ref{fig:oeis:notebook:pascal}; in particular, we take advantage of
(i)~hyper-references to make sequences labels clickable to quickly visit them
in a new tab, (ii)~font styles to emphasize words in italics and
(iii)~bold-face and to render math notations properly such as $2$-dimensional
array representation for matrices.
\vfill

\begin{figure}
\includegraphics{OEIS/notebook-fibonacci}
\caption{This screenshot shows search results about the Fibonacci numbers where
(i)~the section about comments is filtered such that the word "\emph{binomial}"
has to appear in their text and (ii)~the section about formulae is hidden.}
\label{fig:oeis:notebook:fibonacci}
\end{figure}

\begin{figure}
\includegraphics{OEIS/notebook-catalan}
\caption{This screenshot shows search results of a query using a subsequence,
showing \emph{data} sections only.}
\label{fig:oeis:notebook:catalan}
\end{figure}

\begin{figure}
\includegraphics{OEIS/notebook-pascal}
\caption{This screenshot shows search results of an open query using the
"\emph{pascal}" keyword, representing the \emph{data} section as a
$2$-dimensional array.  }
\label{fig:oeis:notebook:pascal}
\end{figure}

\section{The Grapher}

The script \verb|graphing.py| allows us to represent networks where vertices
are sequences and edges are connections among them, according to \verb|xref|
sections in their JSON encodings. It integrates with the crawler tool by
parsing the fetched files and creates \verb|Graph| objects, defined in the
Python module \verb|networkx|, having different layouts according to a set of
drawing algorithms.

It presents a help message to explain itself:
\VerbatimInput[fontsize=\small]{OEIS/graphing-help.txt}

\begin{example}
The following command draws the graph shown in Figure
\ref{fig:oeis:sequences:network}, where the width of each vertex grows
according to the number of its \textit{incoming} connections,
\begin{Verbatim}[fontsize=\small]
$ python3.6 graphing.py --layout FRUCHTERMAN-REINGOLD graph.png
\end{Verbatim}
in order to emphasize most referenced sequences.
\end{example}

\begin{figure}
\includegraphics{OEIS/graph1}
\caption{Sequences network where vertices are emphasized according to the
number of incoming connections.}
\label{fig:oeis:sequences:network}
\end{figure}

Moreover, it can extract essential data from the whole set of JSON files, such
as the list of vertices and edges, to interface with third-party software tools
that provide different visualizations; in particular, libraries using the
\textit{Javascript} programming language are very powerful and the output they
produce are very expressive. For our purposes, we use the \verb|arborjs|
library (freely available at \url{http://arborjs.org/}) to display two
additional graphs described in the next two examples, respectively.

\begin{example}
Figure \ref{fig:oeis:sequences:network:fibonacci:catalan} reports a new
unlabeled graph that shows the underlying structure of sequences connections.
Here, the layout spreads vertices such that the ones having many
\textit{outgoing} connections are centered, while those having poor
connectivity are left on borders.

Under the hood, the Fibonacci and Catalan numbers are the two central sequences
and both of them have an orbit which contains a set of highly connected
sequences.
\end{example}

\begin{figure}
%\begin{sideways}
\includegraphics[width=15cm, height=15cm]{OEIS/points}
\caption{Sequences network abstracting over identifier to spot the underlying
structure.}
%\end{sideways}
\label{fig:oeis:sequences:network:fibonacci:catalan}
\end{figure}

\begin{example}
On the other hand, Figure
\ref{fig:oeis:sequences:network:fibonacci:catalan:labeled} adds labels and
colors to vertices in order to spot their identity and their relevance
according to a combination of their properties. In particular, each color is
represented by an RGB tuple that gets weigths (i)~the number of comments and
formulae for \textit{red}, (ii)~the number of references and links for
\textit{green} and (iii)~the number of incoming and outgoing connections for
\textit{blue}, respectively. Moreover, we get the complement to $255$ of each
component because many sequences have not so many details and this manipulation
allows us to obtain cleaner and more expressive graphs.

For the sake of clarity, the two sequences in evidence are the Fibonacci and
Catalan numbers, the former has the color $(006100)_{16}$ and its complement
$(FFFFFF)_{16}-(006100)_{16}=(FF9EFF)_{16}$ means that it has many comments,
formulae and connections; the latter has the color $(7C00E5)_{16}$ and its
complement $(FFFFFF)_{16}-(7C00E5)_{16}=(83FF1A)_{16}$ means that it has lots
of comments, links and references.
\end{example}

\begin{remark}
Recall that the interpretations given in the previous examples concern a
\textit{subset} of the OEIS only, in particular the one fetched in our session;
finally, the more we crawl, the more graphs are effective and accurate.
\end{remark}

\begin{figure}
\centering
\begin{sideways}
%\includegraphics[width=20cm, height=20cm]{OEIS/labels}
\includegraphics[width=20cm, height=35cm]{OEIS/coloured}
\end{sideways}
\caption{Sequences network with labelel vertices, here we see that the sequence
of \textit{Fibonacci numbers} (\url{https://oeis.org/A000045}) and of
\textit{Catalan numbers} (\url{https://oeis.org/A000108}) are the two central
sequences, respectively.}
\label{fig:oeis:sequences:network:fibonacci:catalan:labeled}
\end{figure}

\begin{example}
Finally, crawling for a while to get more sequences, we represent their
connections in Figure
\ref{fig:oeis:sequences:network:fibonacci:catalan:circular}, arranging them
using a circular layout and we emphasize vertices in the \textit{dominating
set} using the \textit{red} color.
\end{example}

\begin{figure}
\hspace{-2cm}
\includegraphics[width=20cm, height=20cm]{OEIS/circular}
\caption{A bigger sequences network composed of $419$ sequences; here the
sequence of \textit{Fibonacci numbers} is denoted by $\alpha$ and the sequence
of \textit{Catalan numbers} is denoted by $\beta$, respectively.}
\label{fig:oeis:sequences:network:fibonacci:catalan:circular}
\end{figure}

\section*{Conclusions}

This paper presents a suite of tools that interacts with the \textit{Online
Encyclopedia of Integer Sequences}, whose primary goal is to automate simple
and repetitive operations such as (i)~crawling sequences to hold a local copy
stored in JSON files, (ii)~pretty printing data with filtering capabilities,
both in the terminal and in Jupyter (\url{http://jupyter.org/}) notebooks and
(iii)~to visualize connections among sequences using graphs.

In parallel, this suite has been though to be open to extension
and to interface with the hosting environment, UNIX in particular. For
instance, the printer can be used in pipe with the \verb|less| command to gain
scroll and search features for free or the grapher can be augmented to generate
more detailed graph descriptions to be processed by visualization tools.

An additional work direction is to make graphs interactive, namely to tie
together the crawler and the grapher in a web-browser interface such that a
click on a vertex triggers the execution of the fetching process (unless it has
been downloaded already) and the new connections are added to the network
dynamically.

We wish to point out that the suite of tools presented in this work, the
\textit{grapher} in particular, could be used to mine the graph structures
for study regularities and patterns among sequences which looks an interesting
research activity.



\iffalse

\begin{figure}
\includegraphics{OEIS/fibonacci-catalan}
\caption{Sequences network fetched by commands issued in the discussed session.}
\label{fig:oeis:sequences:network}
\end{figure}

\notbreakable{
    \inputminted[fontsize=\small,stripnl=false,firstline=31,lastline=44]
        {python}{../src/graphing.py}
}

\notbreakable{
    \inputminted[fontsize=\small,stripnl=false,firstline=46,lastline=76]
        {python}{../src/graphing.py}
}

\fi
